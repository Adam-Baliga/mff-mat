\documentclass{article}
\usepackage{amsmath}
\usepackage{amssymb}

\begin{document}
\section{Úloha 1}
Pro která $a \in \mathbb{R}$ je zobrazení $f_a: \mathbb{R}^2 \rightarrow \mathbb{R}^2$ surjektivní? 
$$
f_a(x, y)=((a+6) x+a y, a x+y)
$$
\section{Úloha 2}
(1.2) Označme $O_p: \mathbb{R}^2 \rightarrow \mathbb{R}^2$ osovou symetrii podle přímky $p$ a $O_q: \mathbb{R}^2 \rightarrow \mathbb{R}^2$ osovou symetrii podle přímky $q$.

$$
p=\{(0,1)+t(1,0): t \in \mathbb{R}\}, \quad q=\left\{(x, y) \in \mathbb{R}^2: x+y=2\right\}
$$

(a) Najděte obraz bodu $(x, y)$ při zobrazení $O_p$ a při zobrazení $O_q$. (Zde stačí geometrický argument.) 


(b) Najděte obraz bodu $(x, y)$ při složených zobrazení $O_q \circ O_p$ a $O_p \circ O_q$.
\end{document}