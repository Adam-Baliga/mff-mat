\documentclass{article}
\usepackage{amsmath,amssymb,amsthm}

\theoremstyle{definition}
\theoremstyle{plain}
\newtheorem{theorem}{Veta}
\newtheorem{lemma}{Lemma}
\renewcommand{\proofname}{Dôkaz}
\newcommand{\R}{\mathbb{R}}
\newcommand{\N}{\mathbb{N}}
\newcommand{\Z}{\mathbb{Z}}
\newcommand{\fallingfactorial}[2]{#1^{\underline{#2}}}
\newcommand{\vker}{\text{Ker\hspace{0.1em}}}
\newcommand{\vim}{\text{Im\hspace{0.1em}}}
\newcommand{\LO}{\text{LO\hspace{0.1em}}}
\newcommand{\LK}{\text{LK\hspace{0.1em}}}

\newenvironment{solution}{\noindent\textbf{Riešenie.}\hspace{0.5em}}{\hfill\qed\medskip}

\begin{document}
\begin{solution}
Pre ľubovoľný prvok $f\in W$ platí,že je prvok $V$ a teda ho možno zapísať ako $f(x) = ax^3+bx^2+cx+d$ pre $a,b,c,d\in\R$ zároveň $f(2)=0$ a z toho dostneme rovnicu $8a+4b+2c+d=0$ čiže 
$f(x)=ax^3+bx^2+cx -8a-4b-2c$ zoskupením členov dostaneme $f(x)=a(x^3-8)+b(x^2-4)+c(x-2)$. Z toho vyplýva, že množina $\{x^3-8,x^2-4,x-2\}$ generuje prostor $W$ a zároveň posloupnost $(x^3-8,x^2-4,x-2)$ je lineárne nezávislá keďže každý člen posloupnosti polynóm rôzneho stupňa Takže môžeme povedať,že táto posloupnost je bázou $W$
\end{solution}


\end{document}

